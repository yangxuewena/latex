\documentclass{article}
\usepackage[top=0.5in, bottom=1in, left=1in, right=1in]{geometry}
\usepackage{graphicx}
\bibliographystyle{plain}
\usepackage{indentfirst}
\setlength{\parindent}{2em}
\author{Xuewen Yang}
\date{may 14. 2018}
\title{Electronic Imaging}
\begin{document}
\maketitle
Although the film camera opened up a new revolution in underwater photography, an area that was still badly in need of technical advance was the capability to view underwater images, taken from a remote platform, from a surface vessel.To achieve this goal, Ballard's group at WHOI developed the ARGO system Figure~\ref{fig:1}~\cite{Harris1986ARGO}, a substantial advance for deep-sea underwater imaging. The system consisted of a towed sled, much like the acoustically navigated geological underwater survey (ANGUS), but in addition to the film cameras the vehicle was equipped with video cameras that permitted real-time viewing via a cable. The most famous image produced by ARGO, shown in Figure~\ref{fig:2}, was a picture of one of the giant boilers from the sunken luxury liner Titanic, confirming that the hull was nearby. The real-time data proved indispensible in making adaptive decisions about seafloor mapping strategy.
\begin{figure}[htbp]
\centering
\includegraphics[width=6cm]{modernimaging1}
\caption{(a) ARGO video sled}
\label{fig:1}
\end{figure}

\begin{figure}[htbp]
\centering
\includegraphics[width=6cm]{modernimaging2}
\caption{(b) Image of one of the boilers of the sunken luxury liner Titanic, taken in 1985 (Woods Hole Photo Credit).}
\label{fig:2}
\end{figure}

Last, we present a note about the history of the study of the physics of underwater image formation and its use in optimizing the configuration of underwater optical imaging system performance.During the 1960s, a center of excellence in underwater optical imaging was first established at the Massachusetts Institute
of Technology (MIT, Cambridge, MA, USA) and then moved to the Scrips Institution of Oceanography. Under the direction of S. Q. Duntley, the lab spent several decades both refining the limits of, and working on new technology to improve underwater imaging until it was disbanded in the 1980s.Duntley's seminal paper ��Light in the sea�� \cite{Morrison1970Characteristics}, published in 1962, highlights the fact that ��the long research program, spanning two decades,�� on underwater optical imaging by the U.S.Government actually started during World War II. At the same time, largely unbeknownst to the west, work in the former Union of Soviet Socialist Republics (U.S.S.R.) progressed mostly on a theoretical basis, however, with some experimental work. Many references to work in radiative transfer as well as the theory
of optical imaging in scattering media are referenced by Zege et al. in \cite{Zege2011Image}. A third center of excellence was established in Denmark, largely through the work of Jerlov and colleagues.

\bibliography{test3}
\end{document}
