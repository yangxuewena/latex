\documentclass[10pt,twocolumn,letterpaper]{article}

\usepackage{cvpr}
\usepackage{times}
\usepackage{epsfig}
\usepackage{graphicx}
\usepackage{amsmath}
\usepackage{amssymb}

\usepackage{times}


\cvprfinalcopy % *** Uncomment this line for the final submission
\def\cvprPaperID{****} % *** Enter the CVPR Paper ID here
\def\httilde{\mbox{\tt\raisebox{-.5ex}{\symbol{126}}}}

\usepackage{indentfirst}
\setlength{\parindent}{2em}
\usepackage{cite}
\usepackage[colorlinks,linkcolor=red,anchorcolor=blue,citecolor=green,backref=page]{hyperref}

\author{Xuewen Yang\\\\
June 18 2018}

\title{Understanding Image Virality}
\begin{document}
\maketitle
\begin{abstract}
Virality of online content on social networking websites
is an important but esoteric phenomenon often studied in
fields like marketing, psychology and data mining. In this
paper we study viral images from a computer vision perspective.
We introduce three new image datasets from Reddit1
and define a virality score using Reddit metadata. We
train classifiers with state-of-the-art image features to predict
virality of individual images, relative virality in pairs
of images, and the dominant topic of a viral image. We also
compare machine performance to human performance on
these tasks. We find that computers perform poorly with low
level features, and high level information is critical for predicting
virality. We encode semantic information through
relative attributes. We identify the 5 key visual attributes
that correlate with virality. We create an attribute-based
characterization of images that can predict relative virality
with 68:10\% accuracy (SVM+Deep Relative Attributes)
�Cbetter than humans at 60:12\%. Finally, we study how human
prediction of image virality varies with different ��contexts��
in which the images are viewed, such as the influence
of neighbouring images, images recently viewed, as well as
the image title or caption. This work is a first step in understanding
the complex but important phenomenon of image
virality. Our datasets and annotations will be made publicly
available.
\end{abstract}
\section{Introduction}
What graphic should I use to make a new startup more
eye-catching than Instagram? Which image caption will
help spread an under-represented shocking news? Should
I put an image of a cat in my YouTube video if I want millions
of views? These questions plague professionals and
regular internet users on a daily basis. Impact of advertisements,
marketing strategies, political campaigns, non-profit
organizations, social causes, authors and photographers, to
name a few, hinges on their ability to reach and be noticed by a large number of people. Understanding what makes
content viral has thus been studied extensively by marketing
researchers\cite{Berger2011What}.

Many factors such as the time of day and day of week
when the image was uploaded, the title used with the image,
etc. affect whether an image goes viral or not. To
what extent is virality dependent on these external factors,
and how much of the virality depends on the image content
itself? How well can state-of-the-art computer vision
image features and humans predict virality? Which visual
attributes correlate with image virality?

In this paper, we address these questions. We introduce
three image databases collected from Reddit and a virality
score. Our work identifies several interesting directions for
deeper investigation where computer vision techniques can
be brought to bear on this complex problem of understanding
and predicting image virality.
\section{Related Work}
Most existing works\cite{barabasi2005the} study how people share
content on social networking sites after it has been posted.
They use the network dynamics soon after the content has
been posted to detect an oncoming snowballing effect and
predict whether the content will go viral or not. We argue
that predicting virality after the content has already been
posted is too late in some applications. It is not feasible for graphics designers to ��try out�� various designs to see if
they become viral or not. In this paper, we are interested in
understanding the relations between the content itself (even
before it is posted online) and its potential to be viral.

There exist several qualitative theories of the kinds of
content that are likely to go viral\cite{berger2011arousal}. Only a few
works have quantitatively analyzed content, for instance
Tweets and New York Times articles\cite{berger2012what} to predict their
virality. However, in spite of them being a large part of our
online experience, the connections between content in visual
media and their virality has not been analyzed. This
forms the focus of our work.

Virality of text data such as Tweets has been studied in\cite{Nagarajan2010A}. The diffusion properties were found to be dependent
on their content and features like embedded URL��s
and hashtags. Generally, diffusion of content over networks
has been studied more than the causes. The work of
Leskovec \emph{et al.} models propagation of recommendations
over a network of individuals through a stochastic
model, while Beutel \emph{et al.} approach viral diffusion as
an epidemiological problem.
\begin{figure}
\centering
\includegraphics[width=8cm,height=5cm]{1}
\caption{Examples of temporal contextual priming through blurring
in viral images. Looking at the images on the left in both (a)
and (b), what do you think the actual images depict? Did your
expectations of the images turn out to be accurate?}
\label{fig:1}
\end{figure}
\section{Understanding Image Virality}
Consider the viral images of Figure~\ref{fig:1}, where face swapping, contextual priming, and scene gist make
the images quite different from what we might expect at
a first glance. An analogous scenario researched in NLP
is understanding the semantics of ��That��s what she said!��
jokes. We hypothesize that perhaps images that do not
present such a visual challenge or contradiction �C where semantic
perception of an image does not change significantly
on closer examination of the image �C are ��boring�� \cite{berger2012what}
and less likely to be viral. This contradiction need not stem
from the objects or attributes within the image, but may also
rise from the context of the image: be it the images surrounding
an image, or the images viewed before the image,
or the title of the image, and so on. Perhaps an interplay
between these different contexts and resultant inconsistent
interpretations of the image is necessary to simulate a visual
double entendre leading to image virality. With this in
mind, we define four forms of context that we will study to
explore image virality.(1)Intrinsic context: This refers to visual content that is
intrinsic to the pixels of the image.(2) Vicinity context: This refers to the visual content of
images surrounding the image (spatial vicinity).(3) Temporal context: This refers to the visual content of
images seen before the image (temporal vicinity).(4) Textual context: This non-visual context refers to the
title or caption of the image. These titles can sometimes
manifest themselves as visual content (e.g. if it
is photoshopped). A word graffiti has both textual and
intrinsic context, and will require NLP and Computer
Vision for understanding.


{\small
\bibliographystyle{ieee}
\bibliography{1}
}

\end{document}