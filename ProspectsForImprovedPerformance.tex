\documentclass[30pt,twocolumn,letterpaper]{article}

\usepackage{cvpr}
\usepackage{times}
\usepackage{epsfig}
\usepackage{graphicx}
\usepackage{amsmath}
\usepackage{amssymb}

% Include other packages here, before hyperref.

% If you comment hyperref and then uncomment it, you should delete
% egpaper.aux before re-running latex.  (Or just hit 'q' on the first latex
% run, let it finish, and you should be clear).
%\usepackage[pagebackref=true,breaklinks=true,letterpaper=true,colorlinks,bookmarks=false]{hyperref}
%\usepackage{hyperref}\hypersetup{backref,pdfpagemode=FullScreen,colorlinks=true}
\cvprfinalcopy % *** Uncomment this line for the final submission

\def\cvprPaperID{****} % *** Enter the CVPR Paper ID here
\def\httilde{\mbox{\tt\raisebox{-.5ex}{\symbol{126}}}}

% Pages are numbered in submission mode, and unnumbered in camera-ready
%\ifcvprfinal\pagestyle{empty}\fi



\usepackage{graphicx}
\usepackage{indentfirst}
\setlength{\parindent}{2em}
\usepackage{cite}
\usepackage[colorlinks,linkcolor=red,anchorcolor=blue,citecolor=green,backref=page]{hyperref}
%\bibliographystyle{plain}
\author{Xuewen Yang}
\date{May 29. 2018}
\title{Prospects for Improved Performance}
\begin{document}
\maketitle
\section{Coherent Imaging}
Among the various techniques for
imaging through a scattering medium, coherent techniques
have been shown to offer advantages, especially in biomedical
imaging where optical coherence tomography (OCT) has become
a standard method for viewing various classes of objects.
The central idea with OCT is that signal enhancement will result
from processing a complex waveform that is propagated into
the medium and then detected using heterodyne techniques.

In one of the earliest examples of coherent imaging through
a scattering environment, Stetson \cite{Stetson1967Holographic} imaged a pitcher in a
small tank filled with water with suspended particles. Although
no pictures were shown or analyzed quantitatively, the author
claimed that via simple viewing, the use of coherent methods
yielded better images than those that did not. In other work, to
measure the spatial distances over which light can propagate
and still remain coherent, Stachnik \cite{Stachnik1978The} summarily stated that
��coherent behavior is observable in coastal Atlantic waters at a
range of 7.84 m (4 attenuation lengths) and that the interference
patterns that form have good linearity, but can vary significantly
in contrast.��
\section{Biomedical Perspective}
The use of optical biomedical imaging is an area
that has seen tremendous growth in recent decades. Comparing
the difficulty of imaging through living tissue with that of the
ocean leads to the conclusion that the underwater imagers are
quite fortunate, although that is not necessarily their perspective.
A review paper \cite{Dunsby2003Techniques} presents a wealth of information in
addition to offering a somewhat different perspective on much
the same physics. A document that considers the various mathematical
formulations for diffusion imaging is \cite{Arridge2012Optical}. Diffuse
imaging plays an important role in the biomedical case because
the absorption can range between 0.03 and 0.007 1/cm and the
mean free path for scattering can be on the order of 100 m,
leading to a ratio of scattering to absorption of at least 2000.
\section{Advanced Methodologies}
In this section, a number of
more advanced technologies are considered to aid in underwater
imaging. Topics range from the ultimate limitations imposed by
the physics to the use of more advanced hardware and software
for processing. A natural question that should be asked by those
seeking to image underwater is: What is the ultimate limit to
the process? As considered, the answer to this question can be
found in considering the contrast transmittance, as given in several
examples. To date, the standard assumption has been that
several hundred photons are needed per pixel to form an acceptable
image.One can compute, via simple analysis, a limit
to visibility for a scanned system that images one pixel at a time.
So, given approximately 200 photons of shot noise, the ultimate
limit would imply an SNR. Since 1 W of
532-nm green light corresponds to $2.6747\times10^{18}$ photons, the
equation that governs the SNR for a single pixel, assuming that
only ballistic photons are used in the image process, is
\begin{equation}
200=I\times2.67\times{10^18}\times e^{-2al}\longrightarrow37+ln(I)=2\times al
\end{equation}

The results contained in Table\ref{tab} indicate that systems, considering
only absorption-based imaging and assuming 100\% collection
of the reflected photons, should be able to image at close
to 12 attenuation lengths. However, since the VSF for underwater
targets is quite forward pointed, it is not unreasonable
to assume that some fraction of those photons, especially after
reflection from the target, will be useful. Since no underwater
imaging system has achieved the performance as listed in this
table, it appears as if there is some room for improvement.
\begin{table}
\begin{center}
\begin{tabular}{|c|c|}
\hline
Input Power(Watts)&Total No.Atten.Lengths\\
\hline
$10^{-6}$ &11.7\\
\hline
$10^{-3}$ &15.1\\
\hline
1&18.6\\
\hline
\end{tabular}\\

\caption{RESULT OF COMPUTING THE NUMBER OF POTENTIALLY ACHIEVABLE ATTENUATION LENGTHS FOR THE ROUND-TRIP IMAGING FROM (5)}
\label{tab}
\end{center}
\end{table}

{\small
\bibliographystyle{ieee}
\bibliography{1}
}

%\bibliography{1}
\end{document}
