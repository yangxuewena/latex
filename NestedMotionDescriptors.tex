\documentclass[10pt,twocolumn,letterpaper]{article}

\usepackage{cvpr}
\usepackage{times}
\usepackage{epsfig}
\usepackage{graphicx}
\usepackage{amsmath}
\usepackage{amssymb}

\usepackage{times}


\cvprfinalcopy % *** Uncomment this line for the final submission
\def\cvprPaperID{****} % *** Enter the CVPR Paper ID here
\def\httilde{\mbox{\tt\raisebox{-.5ex}{\symbol{126}}}}

\usepackage{indentfirst}
\setlength{\parindent}{2em}
\usepackage{cite}
\usepackage[colorlinks,linkcolor=red,anchorcolor=blue,citecolor=green,backref=page]{hyperref}

\author{Xuewen Yang\\\\
June 16 2018}

\title{Nested Motion Descriptors}
\begin{document}
\maketitle
\begin{abstract}
A nested motion descriptor is a spatiotemporal representation
of motion that is invariant to global camera translation,
without requiring an explicit estimate of optical flow
or camera stabilization. This descriptor is a natural spatiotemporal
extension of the nested shape descriptor\cite{Byrne2013Nested}
to the representation of motion. We demonstrate that the
quadrature steerable pyramid can be used to pool phase,
and that pooling phase rather than magnitude provides an
estimate of camera motion. This motion can be removed using
the log-spiral normalization as introduced in the nested
shape descriptor. Furthermore, this structure enables an elegant
visualization of salient motion using the reconstruction
properties of the steerable pyramid. We compare our
descriptor to local motion descriptors, HOG-3D and HOGHOF,
and show improvements on three activity recognition
datasets
\end{abstract}
\section{Introduction}
The problem of activity recognition is a central problem
in video understanding. This problem is concerned with
detecting actions in a subsequence of images, and assigning
this detected activity a unique semantic label. The core
problem of activity recognition is concerned with the representation
of motion, such that the motion representation
captures the informative or meaningful properties of the
activity, and discards irrelevant motions due to camera or
background clutter.
A key challenge of activity recognition is motion representation
in unconstrained video. Classic activity recognition
datasets focused on tens of actions collected with a
static camera of actors performing scripted activities, however
the state-of-the-art has moved to recognition of hundreds
of activities captured with moving cameras of ��activities
in the wild�� \cite{Marszalek2009Actions}. Moving cameras exhibit
unconstrained translation, rotation and zoom, which introduces
motion at every pixel in addition to pixel motion due
to the foreground activity. The motion due to camera movement is not informative for the activity, and has been shown
to strongly affect activity representation performance\cite{Jain2013Better}.
Recent work has focused on motion descriptors that are
invariant to camera motion\cite{Jain2013Better}.
Local spatiotemporal descriptors such as, such as HOG-HOF or HOG-3D\cite{Klaser2008A}, have shown to be a useful
motion representation for activity recognition. However,
these local descriptors are not invariant to dominant camera
motion. Recent work has focused on aggregating these local
motion descriptors into dense trajectories, where optical
flow techniques are used to provide local tracking of each
pixel. Then, the local motion descriptors are constructed using
differences in the flow field, and then are concatenated
along a trajectory for invariance to global motion. However,
these approaches all rely on estimation of the motion field
using optical flow techniques, which have shown to introduce
artifacts into a video stream due to an early commitment
to motion or over-regularization of the motion field,which can corrupts the motion representation.
In this paper, we propose a new family of binary local
motion descriptors called nested motion descriptors. This
descriptor provides a representation of salient motion that
is invariant to global camera motion, without requiring an
explicit optical flow estimate. The key new idea underlying
this descriptor is that appropriate sampling of scaled
and oriented gradients in the complex steerable pyramid exhibits
a phase shift due to camera motion. This phase shift
can be removed by a technique called a log-spiral normalization,
which computes a phase difference in neighboring
scales and positions, resulting in a relative phase where the
absolute global image motion has been removed. This approach
is inspired by phase constancy, component velocity and motion without movement, which uses
phase shifts as a correction for translation without an explicit
motion field estimate.

\section{Related Work}
The literature on motion representation can be decomposed
into approaches focused on local motion descriptors,
mid-level motion descriptors or global activity descriptors.
In this section, we will focus on local motion representations
only, which are most relevant to this paper.
A local motion descriptor is a representation of the local
movement in a scene centered at a single interest point
in a video. Examples of local motion descriptors include
HOG-HOF, cuboid, extended SURF and
HOG-3D\cite{Klaser2008A}. These descriptors construct spatiotemporal
oriented gradient histograms over small spatial and temporal
support, typically limited to tens of pixels spatially, and
a few frames temporally. HOG-HOF computed over a similar sized spatiotemporal
support. Furthermore, recent evaluations have
shown that activity recognition performance is significantly improved by considering dense regular sampling of descriptors, rather than sparse extraction at interest points.

An interesting recent development has been the development
of local motion descriptors that are invariant to dominant
camera motion. A translating, rotating or zooming
camera introduces global pixel motion that is irrelevant to
the motion of the foreground object. Research has observed
that this camera motion introduces a global translation, divergence
or curl into the optical flow field\cite{Jain2013Better}, and removing
the effect of this global motion significantly improves the
representation of foreground motion for activity recognition.
The motion boundary histogram computes
a global motion field from optical flow, then computes local
histograms of derivatives of the flow field. This representation
is sensitive to local changes in the flow field, and insensitive
to global flow. Motion interchange patterns
compute a patch based local correspondence to recover the
motion of a pixel, followed by a trinary representation of
the relative motion of neighboring patches. Finally, dense
trajectories concatenate HOG-HOF and motion
boundary histograms for a tracked sequence of interest
points forming a long term trajectory descriptor. The
improved dense trajectories with fisher vector encoding
is the current state-of-the-art on large datasets for action
recognition.
\begin{figure}[t]
\centering
\includegraphics[width=8cm,height=5cm]{1}
\caption{From nested shape descriptors to nested motion descriptors.
Nested shape descriptors pool oriented and scaled gradients
magnitude which captures the contrast of an edge in an image.
Nested motion descriptors pool relative phase which captures
translation of an edge. Projecting the spatiotemporal structure of
the nested motion descriptor onto a single image will form the
structure of the nested shape descriptor.}
\label{fig:1}
\end{figure}

\section{Nested Motion Descriptors}
A nested motion descriptor is a representation of salient
motion in a video that is invariant to camera motion. The
nested motion descriptor is an extension of the nested shape
descriptor \cite{Byrne2013Nested} to the representation of motion. Figure\ref{fig:1}
shows that while the nested shape descriptor pools the magnitude
of edges, the nested motion descriptor pools phase
gradients which captures translation of edges in a video. In
this section, we describe this construction in detail.

{\small
\bibliographystyle{ieee}
\bibliography{1}
}

\end{document}