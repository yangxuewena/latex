\documentclass[10pt,twocolumn,letterpaper]{article}

\usepackage{cvpr}
\usepackage{times}
\usepackage{epsfig}
\usepackage{graphicx}
\usepackage{amsmath}
\usepackage{amssymb}

\usepackage{times}


\cvprfinalcopy % *** Uncomment this line for the final submission
\def\cvprPaperID{****} % *** Enter the CVPR Paper ID here
\def\httilde{\mbox{\tt\raisebox{-.5ex}{\symbol{126}}}}

\usepackage{indentfirst}
\setlength{\parindent}{2em}
\usepackage{cite}
\usepackage[colorlinks,linkcolor=red,anchorcolor=blue,citecolor=green,backref=page]{hyperref}

\author{Xuewen Yang\\\\
June 8 2018}

\title{Deep Transfer Metric Learning}
\begin{document}
\maketitle
\begin{abstract}
Conventional metric learning methods usually assume
that the training and test samples are captured in similar
scenarios so that their distributions are assumed to be the
same. This assumption doesn��t hold in many real visual
recognition applications, especially when samples are captured
across different datasets. In this paper, we propose a
new deep transfer metric learning (DTML) method to learn
a set of hierarchical nonlinear transformations for crossdomain
visual recognition by transferring discriminative
knowledge from the labeled source domain to the unlabeled
target domain. Specifically, our DTML learns a deep metric
network by maximizing the inter-class variations and minimizing
the intra-class variations, and minimizing the distribution
divergence between the source domain and the target
domain at the top layer of the network. To better exploit
the discriminative information from the source domain, we
further develop a deeply supervised transfer metric learning
(DSTML) method by including an additional objective
on DTML where the output of both the hidden layers and
the top layer are optimized jointly. Experimental results on
cross-dataset face verification and person re-identification
validate the effectiveness of the proposed methods.
\end{abstract}
\section{Introduction}
How to design a good similarity function plays an important
role in many computer vision and pattern recognition
tasks. Generally, the optimal similarity function for a given
vision problem is task-specific because the underlying data
distributions for different tasks are usually different. Recent
advances in machine learning have shown that learning a
distance metric directly from a set of training examples can
usually achieve proposing performance than hand-crafted
distance metrics\cite{Davis2007Information}\cite{Weinberger2009Distance}. In recent years, a variety of
metric learning algorithms have been proposed in the literature\cite{Hu2014Discriminative}\cite{Weinberger2009Distance}, and some of them have
successfully applied in visual analysis applications such as
face recognition\cite{Hu2014Discriminative}, image classification\cite{Weinberger2009Distance}, human
activity recognition\cite{Du2008Human}, person re-identification\cite{K2012Large}
and visual search.

To this end, in this work, we propose a new deep transfer
metric learning (DTML) method for cross-dataset visual
recognition. Figure~\ref{fig:1} illustrates the basic idea of the proposed
method. Our method learns a set of hierarchical nonlinear
transformations by transferring discriminative knowledge
from the labeled source domain to the unlabeled target
domain, under which the inter-class variations are maximized
and the intra-class variations are minimized, and the
distribution divergence between the source domain and the
target domain at the top layer of the network is minimized,
simultaneously. To better exploit the discriminative information
from the source domain, we further develop a deeply
supervised transfer metric learning (DSTML) method by including
an additional objective on DTML where the output
of both the hidden layers and the top layer are optimized
jointly. Experimental results on cross-dataset face verification
and person re-identification demonstrate the effectiveness
of the proposed methods.
\begin{figure}[htbp]
\centering
\includegraphics[width=8cm,height=4cm]{1}
\caption{The basic idea of the proposed DTML method. For each
sample in the training sets from the source domain and the target
domain, we pass it to the developed deep neural network. We
enforce two constraints on the outputs of all training samples at
the top of the network: 1) the inter-class variations are maximized
and the intra-class variations are minimized, and 2) the distribution
divergence between the source domain and the target domain at the
top layer of the network is minimized.}
\label{fig:1}
\end{figure}
\section{Related Work}
\subsection{Deep Learning}
In recent years, deep learning has attracted
much attention in computer vision and machine
learning due to its superb performance in various tasks.
Generally, deep learning aims to learn hierarchical feature
representations directly from raw data. Recent advances
have shown that deep learning have been successfully
applied to many visual tasks such as image classification, object detection, action recognition\cite{Le2011Learning},
and face recognition. Many deep learning models
have been proposed in recent years, and representative
methods include deep convolutional neural networks,
deep neural networks, deep stacked auto-encoder\cite{Le2011Learning},
deep belief networks, and deeply-supervised nets.
However, most of them aim to learn feature representations
via deep model rather than similarity measure. More recently,
deep learning has also been used in metric learning, and
several metric learning methods have been proposed. For
example, Cai~\emph{et al.} introduced a nonlinear metric learning
method using the stacked independent subspace analysis.
Hu~\emph{et al.} proposed a discriminative deep metric
learning method which employs a conventional neural network
by enforcing a large margin criterion at the top layer
of the network. While these methods have achieved reasonably
good performance, they assume that the training and
test samples are captured in the same environments, which
is not always satisfied in many real applications. In this
work, we proposed a deep transfer metric learning approach
by learning a deep metric network and considering the distribution
difference between the source domain and the target
domain.
\subsection{Transfer Learning}
Transfer learning aims to address
the problem when the distribution of the training data from
the source domain is different from that of the target domain.
Over the past decades, a variety of transfer learning
algorithms have been proposed and they can be
mainly categorized into two classes: instance-based and
feature-based. For the first class, different weights are
learned to rank the training samples in the source domain
for better learning in the target domain. For the second
class, a common feature space is usually learned which can
transfer the information learned from the source domain to
the target domain. In recent years, several transfer learning
techniques have been presented and representative methods
include domain transfer support vector machine,
transfer dimensionality reduction, and transfer metric
learning\cite{Zhang2012Transfer}. While some proposing results can be obtained
by these transfer learning methods, most of them only
consider minimizing the distribution difference between
the source domain and the target domain by using linear
mappings or the kernel trick, which are not effective enough
to transfer the knowledge if the distribution difference is
large and the transfer functions are usually not explicitly
obtained. In this work, we borrow the idea of deep learning
and propose a deep transfer metric learning method by
learning a discriminative distance network with some information
transferred from the source domain.
{\small
\bibliographystyle{ieee}
\bibliography{1}
}

\end{document}
