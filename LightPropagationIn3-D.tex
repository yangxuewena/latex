\documentclass[twocolumn]{article}
\setlength\columnsep{30pt}
\usepackage{ctex}
\usepackage[top=1in, bottom=1in, left=1in, right=1in]{geometry}%ҳ�߾�
\usepackage{graphicx}
\usepackage{indentfirst}
\setlength{\parindent}{2em}
\bibliographystyle{plain}
\title{Light Propagation in 3-D}
\author{Xuewen Yang}
\date{May 11. 2018}
\begin{document}
\maketitle
The fundamental physics of light are well understood and will only be considered briefly in the context of underwater imaging.Modern physics characterizes light as both a wave that can be subject to diffraction and coherent superposition as well as a particle; a photon, as evidenced by the photoelectric effect, is packaged in a quantum. This paper will reflect this duality in referring to both ��photon scatter�� as well as the propagation of the ��light wave.��

The propagation of light follows from a solution of the wave equation as unified by Maxwell in the 1860s. Here, we address only the electric field as conveniently represented by its two orthogonal components $E_x$ and $E_y$ oscillating with phase offset. Envisioning a collection of such propagating waves, the distribution of the resultant electromagnetic field can be characterized using these vectors and their relative synchronization as described by their phase offset. The Stokes vector [I,Q,U,V][total radiation, horizontally linearly polarized, oblique linearly polarized, circularly polarization] characterizes the polarization
��state�� in considering the aggregate of such individual particles or waves. Note that such a description describes the photon or wave propagation in only one propagation direction.

To fully describe the 3-D light field in aqueous media, it is necessary to characterize both directionality and intensity at every location. The radiant light field is the intensity of light at position propagating in a given direction, per differential unit area per differential solid angle. Radiance is measured in W/m -sr. A convenient way to understand radiance is via the use of Gershun's tube\cite{Gershun1939The}, shown in Figure\ref{fig:1}, where a small area measures the radiant flux incident over solid angle that has propagated through the tube that is shown.
\begin{figure}
\centering
\includegraphics[width=8cm]{test1}
\caption{Gershun's tube: Radiance is defined as the flux of optical energy integrated over area that is incident from a cone of solid angle whose boundary is delineated by the tube, as shown.}
\label{fig:1}
\end{figure}
\bibliography{test1}
\end{document}
