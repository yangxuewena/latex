\documentclass[twocolumn]{article}
\setlength\columnsep{30pt}
\usepackage[top=0.5in, bottom=1in, left=1in, right=1in]{geometry}
\usepackage{graphicx}
\usepackage{indentfirst}
\setlength{\parindent}{2em}
\bibliographystyle{plain}
\author{Xuewen Yang}
\date{May 13.2018}
\title{Early Imagers}
\begin{document}
\maketitle
As described in Subsea imaging and vision\cite{Watson2013Subsea}, the first believable report of human underwater imaging was due to Konrad Kyeser, who in 1405 entered the water in his leather suit with flat glass plates to see through into the water. However, as legend has it, Alexander the Great was very interested in seeing underwater. As time went on, a conjecture that he actually had been underwater took hold and many manuscripts documented this historically imagined event. Figure~\ref{fig:1}, reproduced from a German manuscript from 1405,shows Alexander underwater in a glass diving bell with a dog, a cat, and a rooster. The legend recounts that his wife and her lover then cut the chain that was his connection to the surface, leaving him to his own devices to figure out how to escape back to land.Remarkably, he did. Fascination with underwater imaging was clearly present in the middle ages.

\begin{figure}[htbp]
\centering
\includegraphics[width=8cm]{fig1}
\caption{Tempera painting from the 15th century depicting Alexander the Great in a clear diving bell that has been cast adrift by his wife and her lover.}
\label{fig:1}
\end{figure}
On a somewhat more contemporary note, in the 19th century,there was great fascination in exploring the ocean, somewhat akin to our 20th century fascination with space. As such, submersibles were built to carry humans into the shallow depths.Many of these had underwater viewing ports that allowed seeing into perhaps the murky harbors that the submersibles were deployed in. Two interesting examples were the German built submersible ��Sea Devil,�� built in 1856, and the Ictineo I and II, designed and built by the Catalan designer Narcis Monturial in 1861 and 1864. The Sea Devil is credited with taking the first underwater picture, a foggy image of some rocks. Figure~\ref{fig:2} shows a replication of the Ictineo (length equal to 7 m). The ports for viewing are clearly visible in the front, bow, top, and sides.
\begin{figure}[btbp]
\centering
\includegraphics[width=8cm]{fig2}
\caption{Photograph of the replication of the Ictineo I, located at the Marine Museum in Barcelona.}
\label{fig:2}
\end{figure}
\bibliography{test2}
\end{document}
