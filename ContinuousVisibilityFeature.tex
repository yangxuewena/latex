\documentclass[10pt,twocolumn,letterpaper]{article}

\usepackage{cvpr}
\usepackage{times}
\usepackage{epsfig}
\usepackage{graphicx}
\usepackage{amsmath}
\usepackage{amssymb}

\usepackage{times}


\cvprfinalcopy % *** Uncomment this line for the final submission
\def\cvprPaperID{****} % *** Enter the CVPR Paper ID here
\def\httilde{\mbox{\tt\raisebox{-.5ex}{\symbol{126}}}}

\usepackage{indentfirst}
\setlength{\parindent}{2em}
\usepackage{cite}
\usepackage[colorlinks,linkcolor=red,anchorcolor=blue,citecolor=green,backref=page]{hyperref}

\author{Xuewen Yang\\\\
June 4 2018}

\title{Continuous Visibility Feature}
\begin{document}
\maketitle
\begin{abstract}
In this work, we propose a new type of visibility measurement
named Continuous Visibility Feature (CVF). We
say that a point q on the mesh is continuously visible from
another point p if there exists a geodesic path connecting p
and q that is entirely visible by p. In order to efficiently estimate
the continuous visibility for all the vertices in a model,
we propose two approaches that use specific CVF properties
to avoid exhaustive visibility tests. CVF is then measured
as the area of the continuously visible region. With this
stronger visibility measure, we show that CVF better encodes
the surface and part information of mesh than the tradition
line-of-sight based visibility. For example, we show
that existing segmentation algorithms can generate better
segmentation results using CVF and its variants than using
other visibility-based shape descriptors, such as shape
diameter function. Similar to visibility and other mesh surface
features, continuous visibility would have many applications.
\end{abstract}
\section{Introduction}
Feature extraction often serves as a fundamental engineering
task for higher level applications. For example, almost
all of the recognition tasks start with defining or learning
features. This is particularly true for two-dimensional
image that has a simple and uniform grid structure and pixels
can be indexed by a 2D vector in a continuous coordinate
system. Each pixel has fixed number (usually 4 or 8) of
neighbors. Thus most of the image features are defined by
the convolution operations of the local areas. Consequently,
there are some default features like SIFT\cite{Lowe2004Distinctive} and HOG\cite{Dalal2005Histograms}
used widely in computer vision research.
Unlike images, 3D models are usually modeled in the
continuous domain which makes defining 3D features challenging.
Consequently, many of the features defined for 3D
models usually are designed for specific types of shapes.
For example, visibility among points inside a given
shape has been used directly or indirectly as shape features,in particular for the task of semantic shape segmentation.
The intuition behind most of these visibility-based features
is from the observation that two points sampled from the
same semantic part tend to be visible from each other. For
example, shape diameter function (SDF) is a descriptor
to describe the thickness of a mesh defined via local visibility.
The idea is initially proposed for shape segmentation
based on the assumption that a visually meaningful component
should have similar thickness everywhere. However,
this assumption does not hold in many cases. For example,
for a long component, such as a leg, the thickness at one
end of the component may have different thickness from the
other end. Another example is a flat component or a component
whose size grows gradually in a certain direction.
We believe that the traditional line-of-sight visibility
used in all existing visibility-based shape features is insufficient.
In this paper, we will describe a new feature named
continuous visibility feature. The feature provides stronger
visibility measure by considering the continuously visible
region for a vertex or facet. Thus this feature is defined in a
per-vertex manner. We say that a q is continuously visible
by a point p if there exists a geodesic path connecting p and
q that is entirely visible by p. CVF of p is defined as the
area of a set of continuously visible points by p. A more
precise definition of CVF can be found in Sections 3. We
show that CVF better encodes the surface and part information
of mesh than SDF does. Figure~\ref{fig:1} illustrates some
example models color mapped by CVF and SDF values. We
also show that existing segmentation algorithms can generate
better segmentation results using CVF than using other
shape descriptors, such as shape diameter function\cite{Shapira2008Consistent}. We
will demonstrate the segmentation results using the Princeton
Segmentation Benchmark and applications of CVF and
its variants (namely CVFavg, strong CVF and weak CVF)
beyond segmentation.
A major technical challenge of computing CVF is the
computational efficiency because it is known the determining
the visibility of two points is expensive and determining
their continuous visibility will require many more pairwise
visibility checks.We will present two ways
to compute CVF efficiently. The first approach will use the
continuous property of CVF to construct the continuously visible region in a Breadth-First-Search manner. The second
approach further improves the efficiency by first collecting
the potential continuous visible region and then using
visibility test to search the true boundaries of the continuous
visible region. We find that the second approach
provides significant speedup over the first approach.
\begin{figure*}
\centering
\includegraphics{1}
\caption{Examples of SDF and CVF on the same models. Red and blue indicate the largest and smallest SDF and CVF values, respectively.
From these examples, we can see that CVF provides a better metric to distinguish visually different parts of a model. More extensive
comparison can be found}
\label{fig:1}
\end{figure*}

\section{Related Work}
Extracting shape features from 3D model is a fundamental
operation in shape segmentation, matching, retrieval and
even deformation. These features can be roughly classified
into per-vertex/facet features and global signatures. However,
we should point out that some global signatures are
also generated from per-vertex/facet features via statistical
approaches. Osada\emph{et al.} introduced Shape Distributions
to capture the statistical information of vertices�� and
facets�� shape function values. Shape matching is achieved
by measuring the similarity between two shapes�� distribution.
Even though the feature function is simple,
it paves the way for many research works in shape
matching, shape retrieval and other shape analysis tasks.
Features can be further classified into several categories
depending on whether they are invariant to translation,
scaling, and deformation. Some features are designed to
be invariant to rigid transformation, e.g., spin image
and shape context. Invariance to deformation has attracted
more attention. For example, geodesic distance is
used to build deformation invariance features in a multidimensional
scaling based approach and spectral domain
analysis. The idea of heat diffusion process has
triggered the emergence of diffusion geometry. These
features are usually using Laplaca-Beltrami operator, e.g.,
heat kernel signature, global point signature and
multi-solution spectral descriptor.
Most of the aforementioned features are designed for
shape matching, retrieval and correspondence. In shape
segmentation, researchers have developed other types of features. These features directly control the quality of
the final segmentation results for the approaches based on
k-means clustering\cite{Shlafman2002Metamorphosis}, fuzzy clustering, Gaussianmixture
model followed by graph cut\cite{Shapira2008Consistent}. Some other recent
work on segmentation which acquires better result also
needs some feature definition. Liu~\emph{et al.} used
concavity as features. Kalogerakis~\emph{et al.} and Huang proposed the learning-based segmentation using a rich
collection of the features.

Visibility-based Features. Several recent works use visibility
to derive part-aware features. The intuition behind
most of these visibility-based features is that two points in
the a visually or functionally meaningful part (such as the
leg of a table) tend to be visible from each other. For example,
Shape Diameter Function (SDF)\cite{Shapira2008Consistent} captures the
thickness of a shape locally among visible points. SDF is
determined by sampling rays inside the cone in the antinormal
direction of a facet. The SDF value is the sum of
the projected length of the rays inside the model. However,
SDF may not correspond well to a visually meaningful component
if the thickness is not evenly distributed. For example,
as shown in Figure~\ref{fig:1}, the tabletop has quite different
feature values at its center from those on the boundary. In
addition, the feature values at the ends of the leg are also
quite different from those closer to the tabletop.
Another visibility-based feature is Volumetric Shape Image
(VSI). With the motivation of capturing the general
volumetric context instead of only getting the local volumetric
context of the local cone used for sampling rays,
VSI tries to sample rays in more directions. The VSI feature
is computed by first finding the proxy center for each
vertex and then sampling ray at fixed direction. The field
of VSI is achieved by comparing the difference between the
sampling of a source vertex and other vertices on the mesh.
Recently, weak convex decomposition proposed 
also uses lines-of-sights. It first computes the pairwise visibility
between all pairs of the vertices/facets on the mesh.
Then the similarity matrix is constructed with each entry��s value to be 0 or 1 indicating the pairwise invisibility or
visibility, followed by spectral clustering. Even though
the mathematical background seems to be quite straightforward
by referring to the similar issues in machine learning
problem, the decomposition sometimes is not satisfying
especially when a mesh model has a large area of bended
parts. van Kaick~\emph{et al.} used the technique as preprocessing
step to over-segment the mesh, but it also needs
a lot of post-merging.
In all of these features\cite{Shapira2008Consistent}, traditional lineof-
sight visibility is used. However, from the Figure~\ref{fig:2}, we
could see that the continuous visibility makes more sense
than the general visibility in terms of charactering a potential
component.
\begin{figure}
\centering
\includegraphics{2}
\caption{(a) Example of 2D continuous visibility. The vertex r
is continuously visible from the vertex p but the vertex q is not.
(b) While p is continuously visible from q, q is not continuously
visible from p.}
\label{fig:2}
\end{figure}
\section{Definitions and Properties of CVF}
Visibility among the points inside a given shape has been
used directly or indirectly as shape features in the past. The
intuition behind most of these visibility-based features is
from the observation that two points sampled from the same
(visual or functional) part tend to be visible from each other.
It usually remains true if we state the property the other
way around: two points that are visible from each other are
likely to be from the same part. However, there are many
exceptions. For example, in a human model in a standing
pose, a point from the head may see a point in the heel and in
most cases we would distinguish head and heel as different
parts of the model.


{\small
\bibliographystyle{ieee}
\bibliography{1}
}

\end{document}