\documentclass[twocolumn]{article}
\setlength\columnsep{20pt}
\usepackage{ctex}
\usepackage[top=1in, bottom=1in, left=1in, right=1in]{geometry}%ҳ�߾�
\usepackage{graphicx}
\usepackage{indentfirst}
\setlength{\parindent}{2em}
\bibliographystyle{plain}
\title{Underwater Optical Imaging}
\author{Xuewen Yang}
\date{May 9. 2018}
\begin{document}
\maketitle
It is not entirely clear when life originated approximately 3.5�C4 billion years ago, whether it was in the surface of the oceans, hot springs on land, or in the deeper sea. However, what is clear is that the ability to harvest the energy of the sun quickly became the preferred way to survive and proliferate in a nutrient rich world, where the basic building blocks of life could be found and employed for reproduction. In addition, as some of these primordial organisms evolved, their ability increased so that they not only could use the light energy to create carbon-based, macromolecules but also to sense the environment.

Evidence that this is true comes from a retrospective examination of the down welling spectra of water\cite{Land2011Marine} , depicted in Figure~\ref{fig:1}, when compared to the sensitivity of the vertebrate eye as in Figure~\ref{fig:2}.
\begin{figure}[htbp]
\centering
\includegraphics[width=8cm]{cmp1}
\caption{(a) Spectra of daily downward irradiance in seawaters at various depths in the ocean over a broad spectral range}
\label{fig:1}
\end{figure}
\begin{figure}[htbp]
\centering
\includegraphics[width=8cm]{cmp2}
\caption{(b) Normalized sensitivity of the photo receptors in the vertebrate eye: black=rods; blue, green, red=cones}
\label{fig:2}
\end{figure}
 It is clear that the transparency of ocean water provides a ��window of opportunity�� for both harvesting energy and perceiving the world. There is no doubt that it provides the basis for almost all life on Earth.\cite{Jaffe2015Underwater}
\bibliography{test0}
\end{document}
