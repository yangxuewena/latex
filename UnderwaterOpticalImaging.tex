\documentclass[10pt,twocolumn,letterpaper]{article}
\setlength\columnsep{20pt}
\usepackage[top=1in,bottom=1in,left=1in,right=1in]{geometry}
\usepackage{graphicx}
\usepackage{indentfirst}
\bibliographystyle{plain}
\setlength{\parindent}{2em}
\author{Xuewen Yang}
\date{May 25, 2018}
\title{Underwater Optical Imaging: The Past, the Present,and the Prospects}
\begin{document}
\maketitle
\large
\section{Light And The Underwater Environment}
\subsection{Light Propagation in 3-D}
Modern physics characterizes light as both a wave that can be subject to diffraction and coherent superposition as well as a particle; a photon, as evidenced by the photoelectric effect, is packaged in a quantum.The propagation of light follows from a solution of the wave equation as unified by Maxwell in the 1860s.To fully describe the 3-D light field in aqueous media, it is necessary to characterize both directionality and intensity at every location.
\subsection{Transmission of Light and the Physical Properties of the Environment}
It is important to consider the transformations the light incurs as it propagates through the medium.In a real ocean or lake, filled with myriad particles and
properties, so light can be absorbed, refracted, reflected, scattered and depolarized among other phenomena. the most commonly referenced are the attenuation and scattering of light in the more optically transparent window of the electromagnetic spectrum of 400�C700 nm.Total light attenuation per meter consists of
both the absorption and scatter that a photon may suffer when propagating from location to as described by the simple exponential law:
\begin{equation}
I(r)=I(r_0)e^{c(r-r_0)}=I(r_0)e^{-(a+b)(r-r_0)}\label{eq:commutation}
\end{equation}
\section{A Historical Recap Of Underwater Optical Image}
\subsection{Animal Recap Of Underwater Optical Imaging}
Animal vision systems provide a fascinating glimpse into some of the possibilities of underwater imaging.This section discussed some special cases where unique and very interesting adaptations have occurred that meet the challenge of seeing underwater.This includes the use of multiple wavelengths, polarization, and low-light level sensitivity. We also note that to evade predators, a variety of strategies such as transparency and camouflage have been implemented to remain
��cryptic��\cite{Johnsen2000Transparent} so that the organisms are difficult to visually detect.In the case of biological imagers, for the most part, the imager is not illuminating the object of interest, so passive imaging is the main option.Animal underwater vision systems can be characterized in the same way as any imaging system, the most important features being sensitivity, resolution, and contrast. Other interesting features are spectral bandwidth, and the ability to sense polarization.
\subsection{Early Imagers}
Fascination with underwater imaging was clearly present in the middle ages.In the 19th century,there was great fascination in exploring the ocean, somewhat
akin to our 20th century fascination with space.Two interesting examples were the German built submersible ��Sea Devil,�� built in 1856, and the Ictineo I and II,
designed and built by the Catalan designer Narcis Monturial in 1861 and 1864.The Sea Devil is credited with taking the first underwater picture, a foggy image of some rocks. Figure~\ref{fig:1} shows a replication of the Ictineo (length equal to 7 m).The next great advance in underwater imaging accompanied the advent of diving suits.One of the earliest pictures, although
there seems to be some debate\cite{Watson2013Subsea}, was taken by Louis Boutan
who took pictures of either himself or a fellow diver in 1893 at
the French Mediterranean coast at a depth of more than 50 m.
Fig. 8 shows the setup, camera, and one of the pictures taken As reported, the development of arc lamps and then a magnesium powder flash facilitated shorter exposures and allowed Boutan to not have to stand still for 30 min underwater.
\begin{figure*}[t]
\centering
\includegraphics{p1}
\caption{Photograph of the replication of the Ictineo I, located at the Marine Museum in Barcelona.}
\label{fig:l}
\end{figure*}
\subsection{Imaging With Film}
Two major advances in the 20th century facilitated the proliferation
of underwater imaging:

1) the development of the film
camera, starting in the 1890s with George Eastman;

2) the development of the Aqua-Lung in the 1940s by Cousteau and Gagnan.

The first underwater color pictures were reported to
have been taken in 1926�C1927 by William Longley and Charles
Martin off the Florida Keys.

Bruce Mozert developed still and video
housings and underwater flash systems in the 1930s. Haas developed
the popular ��Rolleimarin�� camera in 1943; and in 1957
Cousteau conceived the Calypso camera that in 1961 went on to
become the best known underwater camera of the 20th century
as the ��Nikonos��.
\subsection{Modern Age for Underwater Optical Imaging: Electronic Imaging}
An area that was still badly in need
of technical advance was the capability to view underwater
images, taken from a remote platform, from a surface vessel.
To achieve this goal, Ballard's group at WHOI developed
the ARGO system\cite{Harris1986ARGO},a substantial advance for
deep-sea underwater imaging.Subsequent developments in underwater optical imaging progressed
as the revolution in digital recording, transmission, and
data processing continued.
\section{Underwater Systems: Principle And Design}
As a general classification, considered above, underwater
imaging systems can be regarded as either ��passive�� or ��active,��
the difference being whether the imager creates the light
(active) or not (passive). Although these two systems are different,
the properties of the light field and how it is transformed
during propagation unifies them and, moreover, provides a
common context for understanding underwater optical imaging.
\subsection{Passive Optical Imaging}
The first case of obtaining a quantitative estimate for
the clarity of imaging underwater was obtained from the measurement
of the visibility of the Secchi disk. Created by Angelo
Secchi in 1865, the disk consists of the black and white pattern,
shown in Fig~\ref{fig:2}, printed on a circular disk. The distance
where the disk vanishes, as seen by a human observer, is taken
as the measure of the water's transparency and referred to as
the Secchi depth.A pragmatic solution to predict underwater radiance is the
Hydrolight Model. The model uses the invariant embedding
technique and can accommodate depth-varying IOPs
while providing output such as the radiance and a variety of
AOPs such as the diffuse attenuation coefficient.Validation has
occurred in a number of circumstances. The model
is commercially available (Sequoia Systems, Bellvue WA). As
configured it does not include polarization.
\begin{figure}[t]
\centering
\includegraphics{p2}
\caption{Secchi disk pattern.}
\label{fig:2}
\end{figure}
\subsection{Active Optical Imaging}
As defined, the use of artificial illumination for underwater
viewing is referred to as ��active.�� There is great interest in active
underwater imaging as an improvement in image quality
can be obtained when the illumination is controlled and coordinated
with the receiver.
\bibliography{prospect}
\end{document}
