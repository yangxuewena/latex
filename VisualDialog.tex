\documentclass[10pt,twocolumn,letterpaper]{article}

\usepackage{cvpr}
\usepackage{times}
\usepackage{epsfig}
\usepackage{graphicx}
\usepackage{amsmath}
\usepackage{amssymb}

\usepackage{times}


\cvprfinalcopy % *** Uncomment this line for the final submission
\def\cvprPaperID{****} % *** Enter the CVPR Paper ID here
\def\httilde{\mbox{\tt\raisebox{-.5ex}{\symbol{126}}}}

\usepackage{indentfirst}
\setlength{\parindent}{2em}
\usepackage{cite}
\usepackage[colorlinks,linkcolor=red,anchorcolor=blue,citecolor=green,backref=page]{hyperref}

\author{Xuewen Yang\\\\
June 14 2018}

\title{Visual Dialog}
\begin{document}
\maketitle
\begin{abstract}
We introduce the task of Visual Dialog, which requires an
AI agent to hold a meaningful dialog with humans in natural,
conversational language about visual content. Specifically,
given an image, a dialog history, and a question about
the image, the agent has to ground the question in image,
infer context from history, and answer the question accurately.
Visual Dialog is disentangled enough from a specific
downstream task so as to serve as a general test of machine
intelligence, while being grounded in vision enough to allow
objective evaluation of individual responses and benchmark
progress. We develop a novel two-person chat datacollection
protocol to curate a large-scale Visual Dialog
dataset (VisDial). VisDial contains 1 dialog (10 questionanswer pairs) 
on 140k images from the COCO dataset,
with a total of 1.4M dialog question-answer pairs.
\end{abstract}
\section{Introduction}
We are witnessing unprecedented advances in computer vision
(CV) and artificial intelligence (AI) �C from ��low-level��
AI tasks such as image classification, scene recognition, object detection to ��high-level�� AI tasks such as learning to play Atari video games\cite{Mnih2015Human} and Go,
answering reading comprehension questions by understanding
short stories, and even answering questions
about images and videos.

Despite rapid progress at the intersection of vision and language
�C in particular, in image captioning and visual question
answering (VQA) �C it is clear that we are far from this
grand goal of an AI agent that can ��see�� and ��communicate��.
In captioning, the human-machine interaction consists of
the machine simply talking at the human (��Two people are
in a wheelchair and one is holding a racket��), with no dialog
or input from the human. While VQA takes a significant
step towards human-machine interaction, it still represents
only a single round of a dialog �C unlike in human conversations,
there is no scope for follow-up questions, no memory
in the system of previous questions asked by the user
nor consistency with respect to previous answers provided
by the system
\section{Related Work}
\subsection{Vision and Language}
A number of problems at the intersection
of vision and language have recently gained prominence
�C image captioning\cite{Donahue2017Long}, video/movie
description, text-to-image coreference/grounding, visual storytelling, and of
course, visual question answering (VQA) \cite{Aishwarya2015VQA}. However, all of these involve (at most) a singleshot
natural language interaction �C there is no dialog. Concurrent
with our work, two recent works have also
begun studying this problem of visually-grounded dialog.
\subsection{Visual Turing Test}
Closely related to our work is that of
Geman~\emph{et al.}, who proposed a fairly restrictive ��Visual
Turing Test�� �C a system that asks templated, binary questions.
In comparison, 1) our dataset has free-form, openended
natural language questions collected via two subjects
chatting on Amazon Mechanical Turk (AMT), resulting in
a more realistic and diverse dataset.see figure~\ref{fig:1} 2) The
dataset only contains street scenes, while our dataset
has considerably more variety since it uses images from
COCO. Moreover, our dataset is two orders of magnitude
larger-2591 images in vs~140k images, 10
question-answer pairs per image, total of~1.4M QA pairs.
\begin{figure*}[t]
\centering
\includegraphics[width=15cm,height=6cm]{1}
\caption{Distribution of first n-grams for (left to right) VisDial questions, VQA questions and VisDial answers. Word ordering starts
towards the center and radiates outwards, and arc length is proportional to number of questions containing the word.}
\label{fig:1}
\end{figure*}

\subsection{Text-based Question Answering}
Our work is related
to text-based question answering or ��reading comprehension��
tasks studied in the NLP community. Some recent
large-scale datasets in this domain include the 30M Factoid
Question-Answer corpus, 100K Simple Questions
dataset, DeepMind Q\&A dataset\cite{Hermann2015Teaching}, the 20 artificial
tasks in the bAbI dataset, and the SQuAD dataset for
reading comprehension. VisDial can be viewed as a
fusion of reading comprehension and VQA. In VisDial, the
machine must comprehend the history of the past dialog and
then understand the image to answer the question. By design,
the answer to any question in VisDial is not present in
the past dialog �C if it were, the question would not be asked.
The history of the dialog contextualizes the question �C the
question ��what else is she holding?�� requires a machine to
comprehend the history to realize who the question is talking
about and what has been excluded, and then understand
the image to answer the question.
\subsection{Conversational Modeling and Chatbots}
Visual Dialog is
the visual analogue of text-based dialog and conversation
modeling. While some of the earliest developed chatbots
were rule-based, end-to-end learning based approaches
are now being actively explored. A
recent large-scale conversation dataset is the Ubuntu Dialogue
Corpus, which contains about 500K dialogs extracted
from the Ubuntu channel on Internet Relay Chat
(IRC). Liu~\emph{et al.} perform a study of problems in existing
evaluation protocols for free-form dialog. One important
difference between free-form textual dialog and Vis-
Dial is that in VisDial, the two participants are not symmetric
�C one person (the ��questioner��) asks questions about an
image that they do not see; the other person (the ��answerer��)
sees the image and only answers the questions (in otherwise
unconstrained text, but no counter-questions allowed). This
role assignment gives a sense of purpose to the interaction
(why are we talking? To help the questioner build a mental
model of the image), and allows objective evaluation of
individual responses.

{\small
\bibliographystyle{ieee}
\bibliography{1}
}



\end{document}