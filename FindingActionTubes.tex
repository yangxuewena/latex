\documentclass[10pt,twocolumn,letterpaper]{article}

\usepackage{cvpr}
\usepackage{times}
\usepackage{epsfig}
\usepackage{graphicx}
\usepackage{amsmath}
\usepackage{amssymb}

\usepackage{times}


\cvprfinalcopy % *** Uncomment this line for the final submission
\def\cvprPaperID{****} % *** Enter the CVPR Paper ID here
\def\httilde{\mbox{\tt\raisebox{-.5ex}{\symbol{126}}}}

\usepackage{indentfirst}
\setlength{\parindent}{2em}
\usepackage{cite}
\usepackage[colorlinks,linkcolor=red,anchorcolor=blue,citecolor=green,backref=page]{hyperref}

\author{Xuewen Yang\\\\
June 6 2018}

\title{Finding Action Tubes}
\begin{document}
\maketitle
In object recognition, there are two traditional problems:
whole image classification,"is there a chair in the image?",
and object detection, "is there a chair and where is it in
the image?". The two problems have been quantified by
the PASCAL Visual Object Challenge\cite{Everingham2010The} and more
recently the ImageNet Challenge\cite{Deng2009ImageNet}. The focus has been
on the object detection task due to its direct relationship to
practical, real world applications. When we turn to the field
of action recognition in videos, we find that most work is
focused on video classification,��is there an action present
in the video��, with leading approaches\cite{Simonyan2014Two}\cite{Wang2011Action}\cite{Wang2014Action} trying to
classify the video as a whole. In this work, we address the
problem of action detection, ��is there an action and where
is it in the video��.

Our goal is to build models which can localize and classify
actions in video. Figure~\ref{fig:1} outlines our approach. Inspired
by the recent advances in the field of object detection
in images\cite{Girshick2014Rich}, we start by selecting candidate regions and
use convolutional networks (CNNs) to classify them. Motion
is a valuable cue for action recognition and we utilize
it in two ways. We use motion saliency to eliminate regions 
that are not likely to contain the action. This leads
to a big reduction in the number of regions being processed
and subsequently in compute time. Additionally, we incorporate
kinematic cues to build powerful models for action
detection. Figure~\ref{fig:2} shows the design of our action models.
Given a region, appearance and motion cues are used
with the aid of convolutional neural networks to make a prediction.
Our experiments indicate that appearance and motion
are complementary sources of information and using
both leads to significant improvement in performance. Predictions from all the frames of the video are
linked to produce consistent detections in time. We call the
linked predictions in time action tubes.

Our detection pipeline is inspired by the human vision
system and, in particular, the two-streams hypothesis.
The ventral pathway (��what pathway��) in the visual cortex
responds to shape, color and texture while the dorsal pathway
(��where pathway��) responds to spatial transformations
and movement. We use convolutional neural networks to
computationally simulate the two pathways. The first network,
spatial-CNN, operates on static cues and captures the
appearance of the actor and the environment. The second
network, motion-CNN, operates on motion cues and captures
patterns of movement of the actor and the object (if
any) involved in the action. Both networks are trained to
discriminate between the actors and the background as well
as between actors performing different actions.

We show results on the task of action detection on two
publicly available datasets, that contain actions in real world
scenarios, UCF Sports\cite{Rodriguez2008Action} and J-HMDB. These are
the only datasets suitable for this task, unlike the task of action
classification, where more datasets and of bigger size
(up to 1M videos) exist. Our approach outperforms all
other approaches on UCF sports, with the
biggest gain observed for high overlap thresholds. In particular,
for an overlap threshold of 0.6 our approach shows
a relative improvement of 87.3\%, achieving mean AUC of
41.2\% compared to 22.0\% reported. On the larger
J-HMDB, we present an ablation study and show the effect
of each component when considered separately. Unfortunately,
no other approaches report numbers on this dataset. Additionally, we show that action tubes yield improved results
on action classification on J-HMDB. Using our action
detections we are able to achieve an accuracy of 62.5\% on
J-HMDB, compared to 56.6\% reported by\cite{Wang2011Action} and 56.5\%
achieved by a whole frame video classification technique
with CNNs.

\begin{figure*}
\centering
\includegraphics{1}
\caption{An outline of our approach. (a) Candidate regions are fed into action specific classifiers, which make predictions using static and
motion cues. (b) The regions are linked across frames based on the action predictions and their spatial overlap. Action tubes are produced
for each action and each video.}
\label{fig:1}
\end{figure*}
\begin{figure}
\centering
\includegraphics{2}
\caption{We use action specific SVM classifiers on spatiotemporal
features. The features are extracted from the fc7 layer
of two CNNs, spatial-CNN and motion-CNN, which were trained
to detect actions using static and motion cues, respectively.}
\label{fig:2}
\end{figure}


{\small
\bibliographystyle{ieee}
\bibliography{1}
}

\end{document}
