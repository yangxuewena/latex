\documentclass[12pt]{article}
\usepackage{graphicx}
\usepackage{indentfirst}
\setlength{\parindent}{2em}
\title{How Intelligence Changes As We Age}
\author{Xuewen Yang}
\date{April 29.2018}

\begin{document}

\maketitle

Joshua Hartshorne, an MIT cognitive1 science researcher and the lead author of a study looking at how intelligence changes as we age said that at almost any given age, most of us are getting better at some things and worse at others.
\begin{figure}[htnp]
  \small
  \centering
  \includegraphics[width=9cm]{task}
  \caption{task}
    % \lable{2}
\end{figure}

\section{eighteen years old:}
Scientists use a test called Digit Symbol Substitution to assess everything from dementia to brain damage. It requires people to use a number of cognitive skills at once �� including processing speed, sustained attention, and visual skills. The tool, which typically involves pairing numbers with symbols, is also part of the Wechsler Adult Intelligence Scale, one of the most widely used measures of intelligence.

Hartshorne employed the test in his study of how intelligence changes over time and found that participants' performance generally peaked in their late teens.
\section{twenty-two years old:}
Most adults are bad at memorizing bits of information without context. A classic example of this idea is that you'll have an easier time remembering a story about someone who bakes than a person with the last name Baker. Because there's no context that links the person to the name, it doesn't become firmly lodged in your memory.
\section{thirty-two years old:}
The human brain has a remarkable capacity to recognize and identify faces, and scientists are just beginning to learn why. On average, we know that our ability to learn and remember new faces appears to peak shortly after 30 years old.
\section{forty-three years old:}
A 2015 study from researchers at Harvard University and the Boston Attention and Learning Laboratory suggests that our ability to sustain attention improves with age, reaching its peak around age 43.

\section{fifty years old:}
Many people believe that their math skills go down the drain after they leave school and stop practicing arithmetic. But the next time you try to split up a check, keep this in mind: your ability to do basic subtraction and division doesn't reach its apex until 50years old.

\end{document}
