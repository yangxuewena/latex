\documentclass[twocolumn]{article}
\usepackage{graphicx}
\usepackage{indentfirst}
\setlength{\parindent}{2em}
\usepackage{cite}
\bibliographystyle{plain}
\author{Xuewen Yang}
\date{May 27. 2018}
\title{In Situ Macroscopy and Microscopy}
\begin{document}
\maketitle
\large
In this section, underwater optical imaging of objects in the tens of centimeters to micrometer range is considered. Although the intent is to highlight seagoing systems, some lab systems are described that have potential to transition to at-sea deployments.A somewhat subjective boundary is defined to discern the difference between macroscopy and microscopy. This is a convenient break point, as a large number of systems have been developed for water column imaging of zooplankton that are macroscopes, while there are relatively few in situ microscopes. In both cases, since the viewing distances are relatively short with respect to the absorption and scattering lengths in most open oceanic conditions, the limitations due to these effects are ameliorated. The imaging systems have been applied to characterize both suspended and bottom dwelling organisms and particles in the sea.
\section{Underwater Macroscopy}
In the case of in situ macroscopy, it is likely that the first such system was the shadow graph system developed by Ortner et al. as a silhouette system to take pictures of plankton\cite{Ortner1979Silhouette},\cite{Ortner1981In}. The fundamentals of silhouette photography are described in  where it is demonstrated that a highly collimated beam of light imaged with a high -number camera facilitates a large depth of field.This is useful for keeping subjects in focus over a large volume.Since the pioneering work of Ortner et al., a number of systems have been developed to image a variety of macroscopic pelagic species \cite{Settles2001Schlieren}. Most of these systems employ some variant of the shadow graph or another method known as dark field illumination that projects a conical, or off axis beam, at the specimens so that only scattered light is imaged. This has great advantage for translucent organisms such as many of the plankton in increasing contrast.
\section{Underwater Microscopy}
Underwater microcopy, as defined above, is related to systems that collect images whose resolution is substantially better than 30 m. Such systems naturally
have a small field of view, however, due to the prevalence of microorganisms in the global ecosystem, their characterization is important \cite{Delong2009The}. In addition, the continued and increasing prevalence of harmful algal blooms \cite{Anderson2002Harmful} and their early detection renders this an attractive area for continued technical development.

A somewhat unexplored option, the use of long working distance lenses with LED illumination in a more conventional geometry, has been under development by this author and coworkers for several years now. Although these systems suffer from a short depth of field, the large natural abundance of small particles such as protists permit obtaining many interesting images. As with shadow-graph systems, volumetric estimation of abundance is difficult to obtain without using stereographic methods or some other means for either localizing the target or calibration. Nevertheless, the data are straightforward to obtain and permit the use of various incoherent image modalities that have proved to be quite useful in conventional, land-based, microscopy. Several pictures from a seagoing version of the system are shown in Figure~\ref{fig:1}.
\begin{figure*}[htbp]
\centering
\includegraphics{1}
\caption{Several images of phytoplankton collected at sea with an underwater microscope system.(a)Helical spiral of Eucampia cells, a diatom.(b)Linear
chain of the diatom type Chaetoceros}
\label{fig:1}
\end{figure*}
\bibliographystyle{plain}
\bibliography{1}
\end{document}
