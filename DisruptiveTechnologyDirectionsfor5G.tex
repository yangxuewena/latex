\documentclass[twocolumn]{article}
\usepackage{graphicx}
\usepackage{indentfirst}
\setlength{\parindent}{2em}%��������
\bibliographystyle{plain}
\author{Xuewen Yang}
\date{May 7.2018}
\title{Five Disruptive Technology Directions for 5G}
\begin{document}
\maketitle
5G is coming. What technologies will define it? Will 5G be just an evolution of 4G, or will emerging technologies cause a disruption requiring a wholesale rethinking of entrenched cellular principles? This paper focuses on potential disruptive technologies and their implications for 5G.
We classify the impact of new technologies, leveraging the Henderson-Clark model, as follows:
\begin{itemize}
\item[1)]
Minor changes at both the node and the architectural level, e.g., the introduction of codebooks and signaling support for a higher number of antennas. We refer to these as evolutions in the design.
\item[1)]
Disruptive changes in the design of a class of network nodes, e.g., the introduction of a new waveform. We refer to these as component changes.
\item[1)]
Disruptive changes in the system architecture, e.g., the introduction of new types of nodes or new functions in existing ones. We refer to these as architectural changes.
\item[1)]
Disruptive changes that have an impact at both the node and the architecture levels. We refer to these as radical changes.
\end{itemize}

We focus on disruptive (component, architectural or radical) technologies, driven by our belief that the extremely higher aggregate data rates and the much lower latencies required by 5G cannot be achieved with a mere evolution of the status quo. We believe that the following five potentially disruptive technologies could lead to both architectural and component design changes, as classified in Figure\ref{fig:1}.\cite{Boccardi2014Five}

\begin{figure}[htbp]
\small
\centering
\includegraphics[width=8cm]{5G}
\caption{The five disruptive directions for 5G considered in this paper}
\label{fig:1}
\end{figure}
\bibliography{bibfile2}
\end{document}
