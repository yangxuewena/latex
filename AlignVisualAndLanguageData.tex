\documentclass[10pt,twocolumn,letterpaper]{article}

\usepackage{cvpr}
\usepackage{times}
\usepackage{epsfig}
\usepackage{graphicx}
\usepackage{amsmath}
\usepackage{amssymb}

\usepackage{times}

% Include other packages here, before hyperref.

% If you comment hyperref and then uncomment it, you should delete
% egpaper.aux before re-running latex.  (Or just hit 'q' on the first latex
% run, let it finish, and you should be clear).
%\usepackage[pagebackref=true,breaklinks=true,letterpaper=true,colorlinks,bookmarks=false]{hyperref}
%\usepackage{hyperref}\hypersetup{backref,pdfpagemode=FullScreen,colorlinks=true}
\cvprfinalcopy % *** Uncomment this line for the final submission

\def\cvprPaperID{****} % *** Enter the CVPR Paper ID here
\def\httilde{\mbox{\tt\raisebox{-.5ex}{\symbol{126}}}}

% Pages are numbered in submission mode, and unnumbered in camera-ready
%\ifcvprfinal\pagestyle{empty}\fi

\usepackage{indentfirst}
\setlength{\parindent}{2em}
\usepackage{cite}
\usepackage[colorlinks,linkcolor=red,anchorcolor=blue,citecolor=green,backref=page]{hyperref}

\author{Xuewen Yang\\\\
May 31 2018}

\title{Learning To Align Visual And Language Data}
\begin{document}
\maketitle

Our alignment model assumes an input dataset of images
and their sentence descriptions. Our key insight is that sentences
written by people make frequent references to some
particular, but unknown location in the image. For example,
in Figure~\ref{fig:1},

\begin{figure*}[*htbp]
\centering
\includegraphics[width=0.8\linewidth]{1}
\caption{Overview of our approach. A dataset of images and their sentence descriptions is the input to our model (left).Our model first infers the correspondences (middle) and then learns to generate novel descriptions (right).}
\label{fig:1}
\end{figure*}


the words ��Tabby cat is leaning�� refer to
the cat, the words ��wooden table�� refer to the table, etc.
We would like to infer these latent correspondences, with
the eventual goal of later learning to generate these snippets
from image regions. We build on the approach of Karpathy~\emph{et al.}, who learn to ground dependency tree relations
to image regions with a ranking objective. Our contribution
is in the use of bidirectional recurrent neural network
to compute word representations in the sentence, dispensing
of the need to compute dependency trees and allowing
unbounded interactions of words and their context in the
sentence. We also substantially simplify their objective and
show that both modifications improve ranking performance.



We first describe neural networks that map words and image
regions into a common, multimodal embedding. Then we
introduce our novel objective, which learns the embedding
representations so that semantically similar concepts across
the two modalities occupy nearby regions of the space.


\section{Representing images}
Following prior work , we observe that sentence descriptions
make frequent references to objects and their attributes.
Thus, we follow the method of Girshick~\emph{et al.}
to detect objects in every image with a Region Convolutional
Neural Network (RCNN). The CNN is pre-trained on
ImageNet\cite{Deng2009ImageNet} and finetuned on the 200 classes of the ImageNet
Detection Challenge\cite{Russakovsky2014ImageNet}.
\section{Representing sentences}
To establish the inter-modal relationships, we would like
to represent the words in the sentence in the same hdimensional
embedding space that the image regions occupy.
The simplest approach might be to project every individual
word directly into this embedding. However, this
approach does not consider any ordering and word context
information in the sentence. An extension to this idea is
to use word bigrams, or dependency tree relations as previously
proposed~\cite{Karpathy2014Deep}. However, this still imposes an arbitrary
maximum size of the context window and requires
the use of Dependency Tree Parsers that might be trained on
unrelated text corpora.
\section{Alignment objective}
Since the supervision is at the level of
entire images and sentences, our strategy is to formulate an image-sentence score as a function of the individual region word
scores. Intuitively, a sentence-image pair should have
a high matching score if its words have a confident support
in the image.
\section{Decoding text segment alignments to images}
Consider an image from the training set and its corresponding
sentence. We can interpret the quantity as the unnormalized
log probability of the word describing any of the bounding boxes in the image. However, since we are
ultimately interested in generating snippets of text instead
of single words, we would like to align extended, contiguous
sequences of words to a single bounding box. Note that
the solution that assigns each word independently to
the highest-scoring region is insufficient because it leads to
words getting scattered inconsistently to different regions.
To address this issue, we treat the true alignments as latent
variables in a Markov Random Field (MRF) where the binary
interactions between neighboring words encourage an alignment to the same region.

{\small
\bibliographystyle{ieee}
\bibliography{1}
}

\end{document}
